\chapter{大型电路作品选讲}

\section{迷宫}

\begin{itemize}
	\item 作者:沉睡
	\item 功能:
	\begin{itemize}
		\item 随机生成30*30迷宫(深度优先算法)。
		\item 迷宫的墙通过虚实化构造,且迷宫布局在小地图可见。
	\end{itemize}
	\item 演示视频:\url{https://www.bilibili.com/video/BV1sk4y1m7DE}
	\item 学习本节需要的专业知识:深度优先搜索(Depth-First-Search, DFS)。可能需要的前置知识:图论(Graph Theory)、栈(Stack)。
\end{itemize}

\subsection{生成迷宫的算法}
在我们的模型中,迷宫是一个由正方形房间组成的矩阵,相邻的房间由墙隔开。我们需要打通一些墙,使得所有房间连通,且对于任意两个房间,存在唯一连通它们的路。

我们把整个迷宫看作一个图,其中每个房间是一个节点,相邻房间之间的墙看作一条边。这个图的任意生成树都满足我们的要求,所以我们只需要选取一个生成树算法。在这个作品中,我们使用深度优先搜索构造生成树。

深度优先搜索算法需要一个栈做存储,这个栈的长度需要等于房间的总数才可以保证不溢出。栈中存储的是房间的坐标。同时,还需要一个列表存储每个房间是否已经访问过(即入过栈)。

\begin{remark}
在计算机编程实现这个算法时,我们也可以选择在栈中增加寻路方向的存储以减少反复生成排列的时间。但是在游戏中,这会导致高得多的空间复杂度,进而导致更高的时间复杂度,反而不如反复生成排列划算。
\end{remark}

这个作品使用的算法如下:
\begin{itemize}
	\item 第0步:将左上角房间入栈。
	\item 第1步:取栈顶的房间x。
	\item 第2步:从x上下左右四个房间中随机取一个房间y。若y已被访问过,则跳到第3步;否则打通xy之间的墙,将y入栈,跳到第1步。
	\item 第3步:从x上下左右除y以外的三个房间中随机取一个房间z。若z已被访问过,则跳到第4步;否则打通xz之间的墙,将z入栈,跳到第1步。
	\item 第4步:从x上下左右除yz以外的两个房间中随机取一个房间w。若w已被访问过,则跳到第5步;否则打通xw之间的墙,将w入栈,跳到第1步。
	\item 第5步:取x上下左右除yzw以外的房间t。若t已被访问过,则跳到第6步;否则打通xt之间的墙,将t入栈,跳到第1步。
	\item 第6步:将x出栈。如果栈为空,算法结束;否则跳到第1步。
\end{itemize}

\subsection{数据格式}
迷宫的大小是30*30,用10位二进制可以表示房间坐标,其中前5位是行标,后5位是列标。

\subsection{相邻格的坐标计算}

\subsection{栈}
栈的结构是双向递次电路。因为算法结束的条件是栈为空,所以这个递次不需要复位。

\subsection{访问记录}

\subsection{随机电路}

\subsection{房间设计}